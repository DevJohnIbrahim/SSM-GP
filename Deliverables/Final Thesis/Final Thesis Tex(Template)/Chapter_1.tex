\chapter{Introduction}

Mark Weiser in \textbf{kjhkjsadhkjas }1988 has defined the term ``ubiquitous computing'' as a method of enhancing computer use \cite{abc123} by making many computers available throughout the physical environment, but make them effectively this was discussed by  invisible to the user \cite{Weiser}.

\begin{tabular}{|c|c|c|c|}
\hline  head 1 &  &  &  \\ 
\hline  &  &  &  \\ 
\hline  &  &  &  \\ 
\hline  &  &  &  \\ 
\hline 
\end{tabular}  

\section{Context aware environments}

\begin{equation}
\label{gaboreqn}
Gabor(u,v,\lambda,\theta,\phi,\sigma,\gamma)=e^{-\frac{u'^{2}+
\gamma^{2}v'^{2}}{2\sigma^{2}}}cos(2\pi\frac{u'}{\lambda}+\phi).
\end{equation}



\section{Gesture types}

Object gestures are depending on moving objects carried on hands, head or legs. Figure \ref{fig:IntroTypeOfGestures} (b) shows a sample of object gestures that can be done by mug cup, doors and table tennis rackets.
\textbf{jkhhkj} 

\begin{figure}[tb]
\centering
\includegraphics[width=0.8\textwidth]{images/IntroTypeOfGestures.jpg}
\caption{(a) Hand gestures, (b) Object gestures and (c) Tilt gestures.}
\label{fig:IntroTypeOfGestures}
\end{figure}


